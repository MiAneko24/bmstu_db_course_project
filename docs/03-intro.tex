\section*{ВВЕДЕНИЕ}
\addcontentsline{toc}{section}{ВВЕДЕНИЕ}
Современные системы автоматического управления широко \\*используют методы нечеткой логики. Основная идея экспертных систем заключается в имитации процесса мыслительной деятельности человека и принятия решений на основе некоторых входных данных. Аппарат теории нечетких множеств позволяет моделировать плавный переход от одного множества к другому при определении степени принадлежности числовых параметров, выражать степень уверенности в процессе принятия решений \cite{IncrEfficiency}.

Впервые нечеткие множества были описаны Л. Заде в работе <<Fuzzy sets>>, но наибольшее распространение нечеткая логика получила после доказательства в 1993г. Бартоломеем Коско FAT-теоремы (Fuzzy Approximation Theorem), согласно которой любая математическая система может быть аппроксимирована системой, основанной на нечеткой логике \cite{designadaptive}.

Нечеткие экспертные системы могут состоять из большого количества правил, например, система MYCIN, спроектированная для диагностирования тяжелых бактериальных инфекций и рекомендации лекарств, оперирует базой знаний из 600 правил \cite{MedicineExpSys}. В связи с этим появляется необходимость хранения информации в базе данных.

\textbf{Цель:} Разработать базу данных нечетких экспертных систем и программное обеспечение для работы с ней.

\textbf{Задачи:}
\begin{enumerate}
	\item Провести анализ существующих подходов к хранени..
	\item Спроектировать базу данных экспертных систем.
	\item Разработать программное обеспечение для работы с базой данных.
	\item Исследовать скорость обработки запросов к базе данных при кэшировании и без.
\end{enumerate}

%\textbf{Аннотация:} 
\pagebreak