\section*{ЗАКЛЮЧЕНИЕ}
\addcontentsline{toc}{section}{ЗАКЛЮЧЕНИЕ}
В процессе выполнения курсовой работы были изучены и проанализированы существующие решения проблемы хранения данных о нечетких экспертных системах, выделены их недостатки. По результатам проведенного анализа представлена формулировка решаемой задачи, выделена ролевая модель. Описаны сущности, используемые для хранения информации о нечетких экспертных системах типа Мамдани и Сугено. Описаны алгоритмы, позволяющие получить результат работы экспертной системы, приведена реализация объектов базы данных.

Описан интерфейс программного обеспечения, позволяющего работать с базой данных нечетких экспертных систем. ПО состоит из программы Matlab для добавления в базу данных информации, сгенерированной на основе файла с примерами входных и выходных данных системы, и настольного приложения, позволяющего просматривать и изменять хранимую информацию, подключаться к базе.

Проведено исследование полученных результатов работы системы, по результатам которого выявлено, что $\Pi$-функция, используемая на уровне базы данных, не позволяет полностью аппроксимировать гауссову функцию принадлежности, что приводит к возможности получения некорректного результата работы сгенерированной автоматически системы. В связи с этим одним из направлений дальнейшего развития является добавление поддержки функций \\*нормального распределения на уровне базы данных.

Также исследована скорость выполнения запросов на получение результата работы экспертной системы с кэшированием и без. Выявлено, что использование кэширования дает выигрыш по времени более чем в 600 раз при повторном запросе с неизменными правилами, функциями принадлежности и переменными. При необходимости обновления данных в кэше прогрыш составляет не более 1,1 раза.

Таким образом, поставленные задачи выполнены, цель работы достигнута.
\pagebreak